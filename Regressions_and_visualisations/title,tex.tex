\documentclass[titlepage]{article}
\usepackage[a4paper,margin=1in,footskip=0.25in]{geometry}

\usepackage{lipsum}

\begin{document}

\title{Rights Not Granted. Wrongs Not Righted.}
\author{Vincent Carse \thanks{I am immensely grateful to Dr. Yike Wang, Dr. Matthew Levy, Prof. Martin Pessendorfer, Dr. Judith Shapiro, Ms. Katarzyna Krajniewska, all my EME peers and my friends and family. Without your guidance, insight, support and in particular \textit{patience} with me throughout the past year, this would not have been possible. I'm sure many of you may have found intra-state primary/middle-school education political economy a little dry, but thanks for not showing it too much!}}
\maketitle
\begin{abstract}

This paper\footnote{The 'Rights Not Granted' are the rights to education. The United States' constitution does not directly protect the right to education, nor does it imply any such right. This was the decision reached in San Antonio Independent School District v. Rodriguez (1973), a case which centred on the inequalities which natural result from a school financing system based on local property taxation. In particular it centred around the vastly different tax bases in two neighbouring Texan public school districts and the effect this had on the resources available to students in each district. The decision has stood ever since.
} estimates the educational return to spending in Texan primary and high schools over the ‘02/’03-‘10/’11 period using three methods. I consider the percentage of students passing standardised maths and English tests in grades 5 and 10 as my dependent variables in all models, and each of my panels is balanced. First educational production functions are estimated using a regression with both campus and time fixed effects, with standard errors clustered at the campus level **need to do**. Class size estimates are small but highly statistically significant (p\textless{0.01} to p\textless{0.001}) in all four settings, with estimates of a decrease between 0.544-1.35 in average pass rates per ten extra students. Per pupil expenditure estimates are all significant at the 5\% level ranging a 0.39-1.1 pass rate increase per \$1000 per student. Additionally I fail to reject that teacher salaries and teacher experience are statistically insignificant for maths at grades 5 and 10 and English at grade 10 at the 5\% level **actually test this?**. Two alternate identification strategies are also used to estimate the causal impact of class size on performance. Arbitrary restrictions on tax rates due to the unconstitutionality of state-wide taxes induced some cash-starved districts to max out their tax rates, presenting windows in which their revenue was completely dependent on property values and state aid. Thus, changes in revenue may have been exogenous to changes in school performance. Further to this, random fluctuations  in the oil price provided positive and negative revenue shocks which may constitute a viable instrument for district revenue are also used to estimate class size parameters.

**consistent with literature?**



\end{abstract}

\end{document}



